\documentclass[10pt]{article}
\usepackage[fontsize=10pt]{fontsize}

\usepackage[margin=0.5in]{geometry} 
\usepackage{amsmath,amsthm,amssymb, graphicx, multicol, array}
\usepackage{bbm}

\newcommand{\N}{\mathbb{N}}
\newcommand{\Z}{\mathbb{Z}}
\setcounter{secnumdepth}{0}
\setlength\parindent{0pt}

 
\newenvironment{problem}[2][Problem]{\begin{trivlist}
\item[\hskip \labelsep {\bfseries #1}\hskip \labelsep {\bfseries #2.}]}{\end{trivlist}}

\newenvironment{prelim}[2][Preliminaries]{\begin{trivlist}
\item[\hskip \labelsep {\bfseries #1}\hskip \labelsep {\bfseries #2}]}{\end{trivlist}}
    
\begin{document}
 
\title{6.S091 Problem Set 1}
\author{Suyeol Yun\\
6.S091: Causality}
\maketitle
 
\section{Preliminaries}
\textbf{(a)} 
\begin{align*}
\mathbb{P}_{\mathcal{X}} &= P(U)P(A|U)P(M|A)P(Y|M, U) \\
&= \operatorname{Ber}(0.5) \operatorname{Ber}(U/4) \operatorname{Ber}(0.5 + 0.1A)\operatorname{Ber}(M/2 + U/4)
\end{align*}

\textbf{(b)} 

\begin{align*}
    P(Y|M=0, A=0) &= \mathbbm{1}_{\left\{\varepsilon_y+U / 4 \geq 1\right\}} \sim \operatorname{Ber}(U/4)\\
    P(Y=1|M=0, A=0) &= P(Y=1|M=0, A=0, U=0)P(U=0) \\ &+ P(Y=1|M=0, A=0, U=1)P(U=1)\\
    &= 0 \cdot 1/2 + 1/4 \cdot 1/2\\
    &= 1/8\\
    P(Y=1|M=0, A=1) &= P(Y=1|M=0, A=0)\\
    &= 1/8
\end{align*}


% Since $U, A, M, Y$ are all have a set of alphabets $\{0, 1\}$, they follows Bernoulli distribution.
% Then remaining task is to find the parameter of Bernoulli distribution for each variable.
\begin{align*}
\end{align*}


\begin{align*}
    \mathbb{P}(A=1) &=\sum_{U \in \{0,1\}}\mathbb{P}(A=1|U)P(U) && (\text{Law of Total Probability})\\
    &=\sum_{U \in \{0,1\}}\mathbb{P}(\epsilon_a+U/4\ge1)P(U) \\
    &=\sum_{U \in \{0,1\}}\mathbb{P}(\epsilon_a\ge1-U/4)P(U) \\
    &=\mathbb{P}(\epsilon_a\ge 3/4)P(U=1) + \mathbb{P}(\epsilon_a\ge 1)P(U=0) \\
    &=1/4\cdot1/2 + 0\cdot 1/2\\
    &=1/8
\end{align*}

\begin{align*}
    \mathbb{P}(M=1) &=\sum_{A \in \{0,1\}}\mathbb{P}(M=1|A)P(A) && (\text{Law of Total Probability})\\
    &=\sum_{A \in \{0,1\}}\mathbb{P}(\varepsilon_m+10(1-A) \leq 60)P(A) \\
    &=\sum_{A \in \{0,1\}}\mathbb{P}(\varepsilon_m \leq 60-10(1-A))P(A) \\
    &=\mathbb{P}(\varepsilon_m \leq 60)P(A=1) + \mathbb{P}(\varepsilon_m \leq 50)P(A=0) \\
    &=3/5\cdot1/8 + 1/2\cdot7/8\\
    &=41/80
\end{align*}

\begin{align*}
    \mathbb{P}(Y=1) &=\sum_{M \times U \in \{0,1\} \times \{0,1 \}} \mathbb{P}(Y=1|M, U)P(M, U) && (\text{Law of Total Probability})\\
    &=\sum_{M \times U \in \{0,1\} \times \{0,1 \}} \mathbb{P}(Y=1|M, U)P(M|U)P(U)\\
    &=\sum_{M \in \{0,1\}} \mathbb{P}(Y=1|M, U=0)P(M|U=0)P(U=0) \\ &+ \sum_{M \in \{0,1\}} \mathbb{P}(Y=1|M, U=1)P(M|U=1)P(U=1)\\
    &=\mathbb{P}(Y=1|M=0, U=0)P(M=0|U=0)\cdot 1/2 \\& + \mathbb{P}(Y=1|M=1, U=0)P(M=1|U=0)\cdot 1/2 \\ &+ \mathbb{P}(Y=1|M=0, U=1)P(M=0|U=1)\cdot 1/2 \\ &+\mathbb{P}(Y=1|M=1, U=1)P(M=1|U=1)\cdot 1/2\\
    &=0\cdot 41/80 \cdot 1/2 \\
    & + 1/2 \cdot 39/80 \cdot 1/2 \\ 
    &+ 1/4 \cdot 39/80 \cdot 1/2 \\ 
    &+ 3/4 \cdot 41/80 \cdot 1/2\\
    &=3/8
\end{align*}

In the above calculation, it's worth noting that $U$ nad $M$ are correlated via $A$, i.e. $U \rightarrow A \rightarrow M$. $M$ and $U$ are not independent and thus, $P(M|U)$ is computed as follows.

\begin{align*}
    P(M=0|U=0) &=  P(M=0|U=0, A=0)P(A=0) + P(M=0|U=0, A=1)P(A=1) \\
    &=1/2 \cdot 7/8 + 3/5 \cdot 1/8 \\
    &=41/80\\
P(M=1|U=0) &= 1- 41/80 = 39/80 && \\
P(M=0|U=1) &=  P(M=0|U=1, A=0)P(A=0) + P(M=0|U=1, A=1)P(A=1) \\
&=1/2 \cdot 7/8 + 2/5 \cdot 1/8 \\
&=39/80\\
P(M=1|U=1) &=  41/80 \\
\end{align*}

\section{Interventional}
\subsection{(c)}

\begin{align*}
\mathbb{P}_{\mathcal{X}}(U, A, M, Y \mid \operatorname{do}(A=1)) &=  P(U)P(A^I=1)P(M|A=1)P(Y|M, U)\\
&= \operatorname{Ber}(0.5)\cdot1\cdot\operatorname{Ber}(0.6)\cdot\operatorname{Ber}(M/2 + U/4)\\
&= \operatorname{Ber}(0.5)\cdot\operatorname{Ber}(0.6)\cdot\operatorname{Ber}(M/2 + U/4)
\end{align*}    

\subsection{(d)}
\begin{align*}
    \mathbb{P}_{\mathcal{X}}(Y=1 \mid \text { do }(A=1))  &=\sum_{M \times U \in \{0,1\} \times \{0,1 \}} \mathbb{P}(Y=1|\text { do }(A=1), M, U)P(M, U|\text { do }(A=1)) \\
    &=\sum_{M \times U \in \{0,1\} \times \{0,1 \}} \mathbb{P}(Y=1|\text { do }(A=1), M, U)P(M|\text { do }(A=1))P(U|\text { do }(A=1))\\
    &=\mathbb{P}(Y=1|\text { do }(A=1), M=0, U=0)P(M=0|\text { do }(A=1))P(M|\text { do }(A=1))P(U=0|\text { do }(A=1))
    \\& + \mathbb{P}(Y=1|\text { do }(A=1), M=1, U=0)P(M=1|\text { do }(A=1))P(M|\text { do }(A=1))P(U=0|\text { do }(A=1)) 
    \\&+ \mathbb{P}(Y=1|\text { do }(A=1), M=0, U=1)P(M=0|\text { do }(A=1))P(M|\text { do }(A=1))P(U=1|\text { do }(A=1))
    \\&+\mathbb{P}(Y=1|\text { do }(A=1), M=1, U=1)P(M=1|\text { do }(A=1))P(M|\text { do }(A=1))P(U=1|\text { do }(A=1))\\
    &=0\cdot 2/5\cdot 0 + 1/2 \cdot 3/5\cdot 0  + 1/4 \cdot 2/5\cdot 1 + 0.75 \cdot 3/5\cdot 1\\
    &=11/20\\
    &=0.55\\
    \mathbb{P}_{\mathcal{X}}(Y=0 \mid \text { do }(A=1)) &= 1-0.55 = 0.45
\end{align*}

\begin{align*}
    \mathbb{P}_{\mathcal{X}}(Y=1 \mid \text { do }(A=0))  &=\sum_{M \times U \in \{0,1\} \times \{0,1 \}} \mathbb{P}(Y=1|\text { do }(A=0), M, U)P(M, U|\text { do }(A=0)) \\
    &=\sum_{M \times U \in \{0,1\} \times \{0,1 \}} \mathbb{P}(Y=1|\text { do }(A=0), M, U)P(M|\text { do }(A=0))P(U|\text { do }(A=0))\\
    &=\mathbb{P}(Y=1|\text { do }(A=0), M=0, U=0)P(M=0|\text { do }(A=0))\cdot P(U=0|\text { do }(A=0)) 
    \\& + \mathbb{P}(Y=1|\text { do }(A=0), M=1, U=0)P(M=1|\text { do }(A=0))\cdot P(U=0|\text { do }(A=0)) 
    \\&+ \mathbb{P}(Y=1|\text { do }(A=0), M=0, U=1)P(M=0|\text { do }(A=0))\cdot P(U=1|\text { do }(A=0)) 
    \\&+\mathbb{P}(Y=1|\text { do }(A=0), M=1, U=1)P(M=1|\text { do }(A=0))\cdot P(U=1|\text { do }(A=0))\\
    &=0\cdot 1/2\cdot 4/7 + 1/2 \cdot 1/2\cdot 4/7  + 1/4 \cdot 1/2\cdot 3/7 + 0.75 \cdot 1/2\cdot 3/7\\
    &=5/14\\
    \mathbb{P}_{\mathcal{X}}(Y=0 \mid \text { do }(A=0)) &= 1- 5/14 = 9/14
\end{align*}



\end{document}