\documentclass[10pt]{article}
\usepackage[fontsize=10pt]{fontsize}

\usepackage[margin=0.5in]{geometry} 
\usepackage{amsmath,amsthm,amssymb, graphicx, multicol, array, txfonts}
\usepackage{bbm}
\usepackage{hyperref}
\hypersetup{
    colorlinks=true,
    linkcolor=blue,
    filecolor=magenta,      
    urlcolor=cyan,
    pdftitle={Overleaf Example},
    pdfpagemode=FullScreen,
    }

\urlstyle{same}


\newcommand{\N}{\mathbb{N}}
\newcommand{\Z}{\mathbb{Z}}
\setcounter{secnumdepth}{0}
\setlength\parindent{0pt}

 
\newenvironment{problem}[2][Problem]{\begin{trivlist}
\item[\hskip \labelsep {\bfseries #1}\hskip \labelsep {\bfseries #2.}]}{\end{trivlist}}

\newenvironment{prelim}[2][Preliminaries]{\begin{trivlist}
\item[\hskip \labelsep {\bfseries #1}\hskip \labelsep {\bfseries #2}]}{\end{trivlist}}
    
\begin{document}
 
\title{6.S091 Problem Set 1}
\author{Suyeol Yun\\
6.S091: Causality}
\maketitle
 
\section{Preliminaries}
\subsection{(a)} 
\begin{align*}
\mathbb{P}_{\mathcal{X}}(U, A, M, Y) &= P(U)P(A|U)P(M|A)P(Y|M, U) \\
&= \operatorname{Ber}(0.5) \operatorname{Ber}(U/4) \operatorname{Ber}(0.5 + 0.1A)\operatorname{Ber}(M/2 + U/4)
\end{align*}

\subsection{(b)}
\begin{align*}
&\mathbb{P}_{\mathcal{X}}(Y=1) = 0.38125\\
&\mathbb{P}_{\mathcal{X}}(Y=1 \mid M=0, A=0)= 0.10714285714285714\\
&\mathbb{P}_{\mathcal{X}}(Y=1 \mid M=0, A=1)= 0.25\\
\end{align*}
* I computed the above values by computing the distribution $\mathbb{P}_{\mathcal{X}}(U, A, M, Y)$ as dictionary in python and marginalized. 
See \url{https://github.com/syyunn/6.S091/blob/main/pset1/problem_1b.py}


\section{Interventional}
\subsection{(c)}

\begin{align*}
\mathbb{P}_{\mathcal{X}}(U, A=1, M, Y \mid \operatorname{do}(A=1)) &=  P(U)P(A=1|A=1)P(M|A=1)P(Y|M, U, A=1)\\
&= \operatorname{Ber}(0.5)\cdot1\cdot\operatorname{Ber}(0.6)\cdot\operatorname{Ber}(M/2 + U/4)\\
&= \operatorname{Ber}(0.5)\cdot\operatorname{Ber}(0.6)\cdot\operatorname{Ber}(M/2 + U/4)
\end{align*}    

In the similar vein, we can compute
\begin{align*}
    \mathbb{P}_{\mathcal{X}}(U, A=1, M, Y \mid \operatorname{do}(A=0)) &=  P(U)P(A=0|A=0)P(M|A=0)P(Y|M, U, A=1)\\
    &= \operatorname{Ber}(0.5)\cdot1\cdot\operatorname{Ber}(0.5)\cdot\operatorname{Ber}(M/2 + U/4)\\
    &= \operatorname{Ber}(0.5)\cdot\operatorname{Ber}(0.5)\cdot\operatorname{Ber}(M/2 + U/4)
\end{align*}    
        

\subsection{(d)}
% \begin{align*}
%     \mathbb{P}_{\mathcal{X}}(Y=1 \mid \text { do }(A=1))  &=\sum_{M \times U \in \{0,1\} \times \{0,1 \}} \mathbb{P}(Y=1|\text { do }(A=1), M, U)P(M, U|\text { do }(A=1)) \\
%     &=\sum_{M \times U \in \{0,1\} \times \{0,1 \}} \mathbb{P}(Y=1|\text { do }(A=1), M, U)P(M|\text { do }(A=1))P(U|\text { do }(A=1))\\
%     &=\mathbb{P}(Y=1|\text { do }(A=1), M=0, U=0)P(M=0|\text { do }(A=1))P(M|\text { do }(A=1))P(U=0|\text { do }(A=1))
%     \\& + \mathbb{P}(Y=1|\text { do }(A=1), M=1, U=0)P(M=1|\text { do }(A=1))P(M|\text { do }(A=1))P(U=0|\text { do }(A=1)) 
%     \\&+ \mathbb{P}(Y=1|\text { do }(A=1), M=0, U=1)P(M=0|\text { do }(A=1))P(M|\text { do }(A=1))P(U=1|\text { do }(A=1))
%     \\&+\mathbb{P}(Y=1|\text { do }(A=1), M=1, U=1)P(M=1|\text { do }(A=1))P(M|\text { do }(A=1))P(U=1|\text { do }(A=1))\\
%     &=0\cdot 2/5\cdot 0 + 1/2 \cdot 3/5\cdot 0  + 1/4 \cdot 2/5\cdot 1 + 0.75 \cdot 3/5\cdot 1\\
%     &=11/20\\
%     &=0.55\\
%     \mathbb{P}_{\mathcal{X}}(Y=0 \mid \text { do }(A=1)) &= 1-0.55 = 0.45
% \end{align*}

% \begin{align*}
%     \mathbb{P}_{\mathcal{X}}(Y=1 \mid \text { do }(A=0))  &=\sum_{M \times U \in \{0,1\} \times \{0,1 \}} \mathbb{P}(Y=1|\text { do }(A=0), M, U)P(M, U|\text { do }(A=0)) \\
%     &=\sum_{M \times U \in \{0,1\} \times \{0,1 \}} \mathbb{P}(Y=1|\text { do }(A=0), M, U)P(M|\text { do }(A=0))P(U|\text { do }(A=0))\\
%     &=\mathbb{P}(Y=1|\text { do }(A=0), M=0, U=0)P(M=0|\text { do }(A=0))\cdot P(U=0|\text { do }(A=0)) 
%     \\& + \mathbb{P}(Y=1|\text { do }(A=0), M=1, U=0)P(M=1|\text { do }(A=0))\cdot P(U=0|\text { do }(A=0)) 
%     \\&+ \mathbb{P}(Y=1|\text { do }(A=0), M=0, U=1)P(M=0|\text { do }(A=0))\cdot P(U=1|\text { do }(A=0)) 
%     \\&+\mathbb{P}(Y=1|\text { do }(A=0), M=1, U=1)P(M=1|\text { do }(A=0))\cdot P(U=1|\text { do }(A=0))\\
%     &=0\cdot 1/2\cdot 4/7 + 1/2 \cdot 1/2\cdot 4/7  + 1/4 \cdot 1/2\cdot 3/7 + 0.75 \cdot 1/2\cdot 3/7\\
%     &=5/14\\
%     \mathbb{P}_{\mathcal{X}}(Y=0 \mid \text { do }(A=0)) &= 1- 5/14 = 9/14
% \end{align*}
\begin{equation}
    \mathbb{P}_{\mathcal{X}}(Y \mid \operatorname{do}(A=1))=
    \begin{cases}
        0.425 & \text{if $Y=1$} \\
        0.575 & \text{if $Y=0$}
    \end{cases}
  \end{equation}

  \begin{equation}
    \mathbb{P}_{\mathcal{X}}(Y \mid \operatorname{do}(A=0))=
    \begin{cases}
        0.375 & \text{if $Y=1$} \\
        0.625 & \text{if $Y=0$}
    \end{cases}
  \end{equation}
\\
* I computed the above values by computing the distribution $\mathbb{P}_{\mathcal{X}}(U, A, M, Y)$ as dictionary in python and marginalized. 
See \url{https://github.com/syyunn/6.S091/blob/main/pset1/problem_1d_doA0.py} and \url{https://github.com/syyunn/6.S091/blob/main/pset1/problem_1d_doA1.py} 


\subsection{(e)}
In case of $do(A=0)$, we have
% If $A=1, U=0$, $P(A=1, U=0)=0$.  Therefore, $P(Y|A=1, U=0)$ is undefined. So I check only for the case of $do(A=0)$ which doesn't involve $P(Y|A=1, U=0)$ as a component in its process of computation. \\
\begin{align*}
    \mathbb{P}_{\mathcal{X}}(Y=1 \mid \operatorname{do}(A=0))&=0.375 \quad \text{from (d)}\\
    \sum_{u \in\{0,1\}} \mathbb{P}_{\mathcal{X}}(Y=1 \mid A=0, U=u) \mathbb{P}_{\mathcal{X}}(U=u)&=\mathbb{P}_{\mathcal{X}}(Y=1 \mid A=0, U=1) \mathbb{P}_{\mathcal{X}}(U=1) + \mathbb{P}_{\mathcal{X}}(Y=1 \mid A=0, U=0) \mathbb{P}_{\mathcal{X}}(U=0)\\
    &= 0.5 * 1/2 + 0.25 * 1/2\\
    &= 0.375 \text{\quad (match)}
\end{align*}

\begin{align*}
    \mathbb{P}_{\mathcal{X}}(Y=0 \mid \operatorname{do}(A=0))&=0.625 \quad \text{from (d)}\\
    \sum_{u \in\{0,1\}} \mathbb{P}_{\mathcal{X}}(Y=0 \mid A=0, U=u) \mathbb{P}_{\mathcal{X}}(U=u)&=\mathbb{P}_{\mathcal{X}}(Y=0 \mid A=0, U=1) \mathbb{P}_{\mathcal{X}}(U=1) + \mathbb{P}_{\mathcal{X}}(Y=0 \mid A=0, U=0) \mathbb{P}_{\mathcal{X}}(U=0)\\
    &= 0.5 * 1/2 + 0.75 * 1/2\\
    &= 0.625 \text{\quad (match)}
\end{align*}

In case of $do(A=1)$, we have
% If $A=1, U=0$, $P(A=1, U=0)=0$.  Therefore, $P(Y|A=1, U=0)$ is undefined. So I check only for the case of $do(A=0)$ which doesn't involve $P(Y|A=1, U=0)$ as a component in its process of computation. \\
\begin{align*}
    \mathbb{P}_{\mathcal{X}}(Y=1 \mid \operatorname{do}(A=1))&=0.425 \quad \text{from (d)}\\
    \sum_{u \in\{0,1\}} \mathbb{P}_{\mathcal{X}}(Y=1 \mid A=1, U=u) \mathbb{P}_{\mathcal{X}}(U=u)&=\mathbb{P}_{\mathcal{X}}(Y=1 \mid A=1, U=1) \mathbb{P}_{\mathcal{X}}(U=1) + \mathbb{P}_{\mathcal{X}}(Y=1 \mid A=1, U=0) \mathbb{P}_{\mathcal{X}}(U=0)\\
    &= 0.55 * 1/2 + \text{undefined}\\
    &= \text{undefined}
\end{align*}

\begin{align*}
    \mathbb{P}_{\mathcal{X}}(Y=0 \mid \operatorname{do}(A=1))&=0.625 \quad \text{from (d)}\\
    \sum_{u \in\{0,1\}} \mathbb{P}_{\mathcal{X}}(Y=0 \mid A=1, U=u) \mathbb{P}_{\mathcal{X}}(U=u)&=\mathbb{P}_{\mathcal{X}}(Y=0 \mid A=1, U=1) \mathbb{P}_{\mathcal{X}}(U=1) + \mathbb{P}_{\mathcal{X}}(Y=0 \mid A=1, U=0) \mathbb{P}_{\mathcal{X}}(U=0)\\
    &= 0.45 * 1/2 + \text{undefined}\\
    &= \text{undefined}
\end{align*}


\subsection{(f)}

In case of $do(A=0)$, we have

\begin{align*}
    &\mathbb{P}_{\mathcal{X}}(Y=1 \mid \operatorname{do}(A=0))=0.375 \quad \text{from (d)}\\
    &\sum_{m \in\{0,1\}} \mathbb{P}_{\mathcal{X}}(M=m \mid A=0) \sum_{a^{\prime} \in\{0,1\}}\left(\mathbb{P}_{\mathcal{X}}\left(Y=1 \mid M=m, A=a^{\prime}\right) \mathbb{P}_{\mathcal{X}}\left(A=a^{\prime}\right)\right)\\
    &= \mathbb{P}_{\mathcal{X}}(M=0 \mid A=0) \cdot \sum_{a^{\prime} \in\{0,1\}}\left(\mathbb{P}_{\mathcal{X}}\left(Y=1 \mid M=0, A=a^{\prime}\right) \mathbb{P}_{\mathcal{X}}\left(A=a^{\prime}\right)\right) \\ 
    &+ \mathbb{P}_{\mathcal{X}}(M=1 \mid A=0) \cdot \sum_{a^{\prime} \in\{0,1\}}\left(\mathbb{P}_{\mathcal{X}}\left(Y=1 \mid M=1, A=a^{\prime}\right) \mathbb{P}_{\mathcal{X}}\left(A=a^{\prime}\right)\right) \\
    &= 1/2 \cdot (0.10714285714285714\cdot 0.875 + 0.25 \cdot 0.125) + 1/2 \cdot(0.6071428571428571 \cdot 0.875 + 0.75\cdot 0.125)\\
    &= 0.375 \text{\quad (match)}\\\\
    &\mathbb{P}_{\mathcal{X}}(Y=0 \mid \operatorname{do}(A=0))=0.625 \quad \text{from (c)}\\
    &\sum_{m \in\{0,1\}} \mathbb{P}_{\mathcal{X}}(M=m \mid A=0) \sum_{a^{\prime} \in\{0,1\}}\left(\mathbb{P}_{\mathcal{X}}\left(Y=0 \mid M=m, A=a^{\prime}\right) \mathbb{P}_{\mathcal{X}}\left(A=a^{\prime}\right)\right)\\
    &= \mathbb{P}_{\mathcal{X}}(M=0 \mid A=0) \cdot \sum_{a^{\prime} \in\{0,1\}}\left(\mathbb{P}_{\mathcal{X}}\left(Y=0 \mid M=0, A=a^{\prime}\right) \mathbb{P}_{\mathcal{X}}\left(A=a^{\prime}\right)\right) \\ 
    &+ \mathbb{P}_{\mathcal{X}}(M=1 \mid A=0) \cdot \sum_{a^{\prime} \in\{0,1\}}\left(\mathbb{P}_{\mathcal{X}}\left(Y=0 \mid M=1, A=a^{\prime}\right) \mathbb{P}_{\mathcal{X}}\left(A=a^{\prime}\right)\right) \\
    &= 1/2 \cdot (0.8928571428571429\cdot 0.875 + 0.7500000000000001 \cdot 0.125) + 1/2 \cdot(0.39285714285714285 \cdot 0.875 + 0.25\cdot 0.125)\\
    &= 0.6250000000000000218750 \text{\quad (match)}
\end{align*}

In case of $do(A=1)$, we have

\begin{align*}
    &\mathbb{P}_{\mathcal{X}}(Y=1 \mid \operatorname{do}(A=1))=0.425 \quad \text{from (d)}\\
    &\sum_{m \in\{0,1\}} \mathbb{P}_{\mathcal{X}}(M=m \mid A=1) \sum_{a^{\prime} \in\{0,1\}}\left(\mathbb{P}_{\mathcal{X}}\left(Y=1 \mid M=m, A=a^{\prime}\right) \mathbb{P}_{\mathcal{X}}\left(A=a^{\prime}\right)\right)\\
    &= \mathbb{P}_{\mathcal{X}}(M=0 \mid A=1) \cdot \sum_{a^{\prime} \in\{0,1\}}\left(\mathbb{P}_{\mathcal{X}}\left(Y=1 \mid M=0, A=a^{\prime}\right) \mathbb{P}_{\mathcal{X}}\left(A=a^{\prime}\right)\right) \\ 
    &+ \mathbb{P}_{\mathcal{X}}(M=1 \mid A=1) \cdot \sum_{a^{\prime} \in\{0,1\}}\left(\mathbb{P}_{\mathcal{X}}\left(Y=1 \mid M=1, A=a^{\prime}\right) \mathbb{P}_{\mathcal{X}}\left(A=a^{\prime}\right)\right) \\
    &= 0.4 \cdot (0.10714285714285714\cdot 0.875 + 0.25 \cdot 0.125) + 0.6 \cdot(0.6071428571428571 \cdot 0.875 + 0.75\cdot 0.125)\\
    &= 0.42499999999999997650 \text{\quad (match)}\\\\
    &\mathbb{P}_{\mathcal{X}}(Y=0 \mid \operatorname{do}(A=1))=0.575 \quad \text{from (c)}\\
    &\sum_{m \in\{0,1\}} \mathbb{P}_{\mathcal{X}}(M=m \mid A=1) \sum_{a^{\prime} \in\{0,1\}}\left(\mathbb{P}_{\mathcal{X}}\left(Y=0 \mid M=m, A=a^{\prime}\right) \mathbb{P}_{\mathcal{X}}\left(A=a^{\prime}\right)\right)\\
    &= \mathbb{P}_{\mathcal{X}}(M=0 \mid A=1) \cdot \sum_{a^{\prime} \in\{0,1\}}\left(\mathbb{P}_{\mathcal{X}}\left(Y=0 \mid M=0, A=a^{\prime}\right) \mathbb{P}_{\mathcal{X}}\left(A=a^{\prime}\right)\right) \\ 
    &+ \mathbb{P}_{\mathcal{X}}(M=1 \mid A=1) \cdot \sum_{a^{\prime} \in\{0,1\}}\left(\mathbb{P}_{\mathcal{X}}\left(Y=0 \mid M=1, A=a^{\prime}\right) \mathbb{P}_{\mathcal{X}}\left(A=a^{\prime}\right)\right) \\
    &= 0.4 \cdot (0.8928571428571429\cdot 0.875 + 0.7500000000000001 \cdot 0.125) + 0.6 \cdot(0.39285714285714285 \cdot 0.875 + 0.25\cdot 0.125)\\
    &= 0.57500000000000001625 \text{\quad (match)}
\end{align*}

\section{Problem 2}

\subsection{(a)}

Let's take a arbitrary node $\gamma_m \in \gamma$. Then $\gamma_m$ should be unblocked since $\gamma$ is d-connecting path and it means all nodes in $\gamma$ are unblocked.
Then since $\gamma_m$ is unblocked, we can think of two cases where $\gamma_m$ is collider or $\gamma_m$ is non-collider. First, in case of $\gamma_m$ is collider, then $\gamma_m\in S$ or its descendent $\text{de}_{\mathcal{G}}(\gamma_m) \in S$. If $\gamma_m \in S$, $\gamma_m \in V$ since $V$ include $S$ itself. If $\text{de}_{\mathcal{G}}(\gamma_m) \in S$, $\gamma_m \in \text{an}_\mathcal{G}(S)$, and it implies $\gamma_m \in V$. 
Second, in case of $\gamma_m$ is non-collider, $\gamma_m \notin S$. In this case, let's think about the sub-path $\gamma_m \rightarrow \gamma_{m+1}, \gamma_{m+2}, \cdots, b$ where $b$ is the destination of the path $\gamma$. Among the nodes in this sub-path, if there exists a node that is collider, then this node is in $S$ or ancestor of $S$ as we've shown in the first scenario. Then $\gamma_m$ becomes ancestor of $S$ and it implies $\gamma_m \in V$. 
In other case, if there is no collider in the sub-path, 
then $\gamma_m$ is an ancestor of $b$ and it implies $\gamma_m \in V$.
Therefore, in any cases, $\gamma_m \in V$.

\subsection{(b)}

From $(a)$, we know that $\gamma_m \in S$ only in case that $\gamma_m$ is collider. By the definition of moral graph, if $\gamma_m$ is collider, we can marry 
$\gamma_{m-1}$ and $\gamma_{m+1}$ which are left and right nodes of $\gamma_m$. Therefore, with given path $\gamma$, we can construct a new path $\gamma'$ by visiting each node in $\gamma$ and marry left and right nodes when the node we're visiting is collider. Then this new path $\gamma'$ doesn't pass through $S$ and 
$\gamma'$ is a valid path in moral graph $\overline{\mathcal{G}[\mathbf{V}]}$ since its all nodes and edges are included in $\overline{\mathcal{G}[\mathbf{V}]}$.

\section{Problem 3}

\subsection{(a)}
\begin{align*}
    \mathbb{P}_{\mathcal{X}}(Y=1 \mid A=N) &= \sum_{S \in \{L, R\}}\mathbb{P}_{\mathcal{X}}(Y=1 \mid S, A=N) \mathbb{P}_{\mathcal{X}}(S)\\   
    &= 0.73 \cdot 0.49 + 0.93 \cdot  0.51 \\
    &= 0.832\\
    \mathbb{P}_{\mathcal{X}}(Y=1 \mid A=O) &= \sum_{S \in \{L, R\}}\mathbb{P}_{\mathcal{X}}(Y=1 \mid S, A=O) \mathbb{P}_{\mathcal{X}}(S)\\   
    &= 0.69 \cdot 0.49 + 0.87 \cdot  0.51 \\
    &= 0.7818
\end{align*}

\subsection{(b)}
% (Ask how to draw $\zeta^I$)
Since $S$ satisfies backdoor criterion, 
$\mathbb{P}_{\mathcal{X}}(Y=1 \mid \operatorname{do}(A=N)) = \sum_{\mathbf{s}} \mathbb{P}_{\mathcal{X}}(Y \mid A=N, \mathbf{S}=\mathbf{s}) \mathbb{P}_{\mathcal{X}}(\mathbf{S}=\mathbf{s})=0.832$ and 
$\mathbb{P}_{\mathcal{X}}(Y=1 \mid \operatorname{do}(A=O)) = \sum_{\mathbf{s}} \mathbb{P}_{\mathcal{X}}(Y \mid A=O, \mathbf{S}=\mathbf{s}) \mathbb{P}_{\mathcal{X}}(\mathbf{S}=\mathbf{s})=0.7818$.

\section{Problem 4}

\begin{center}
    \begin{tabular}{||c c c c c||} 
     \hline
     $\beta_{aw}$ & $\hat{\beta}_{aw}$ & $\hat{\beta}_{yw}$ & $\widehat{\beta}_{y w} / \widehat{\beta}_{a w}$ & $|\mathbb{E}[Y|A] - \widehat{\beta}_{y w} / \widehat{\beta}_{a w}|$ \\ [0.5ex] 
     \hline\hline
     5 & 4.52 & 33.84 & 7.47 & 0.03\\ 
     \hline
     0.5 & 0.72 & 5.54 & 7.60 & 0.1\\
     \hline
     0.05 & 0.32 & 2.49 & 7.62 & 0.12\\
     \hline
    \end{tabular}
\end{center}

As shown in the table, the ratio well estimates the true $\mathbb{E}[Y|A]=0.75$. Also we can notice that the absolute value of bias $|\mathbb{E}[Y|A] - \widehat{\beta}_{y w} / \widehat{\beta}_{a w}|$ increases as $\beta_{aw}$ decreases.\\


* I computed the above values by computing the distribution $\mathbb{P}_{\mathcal{X}}(U, A, M, Y)$ as dictionary in python and marginalized. 
See \url{https://github.com/syyunn/6.S091/blob/main/pset1/problem4.py}



% $\beta_{aw}: 5, b_wa_hat: 4.527291388797304, b_wy_hat: 33.847991586594546, ratio: 7.476433187037785,  b_ay: 7.5, error:0.0031422417282952844
% b_wa: 0.5, b_wa_hat: 0.7290866735076191, b_wy_hat: 5.542779314016848, ratio: 7.602359932531293,  b_ay: 7.5, error:0.013647991004172392
% b_wa: 0.05, b_wa_hat: 0.32646743951621543, b_wy_hat: 2.490756958589264, ratio: 7.62941922257319,  b_ay: 7.5, error:0.017255896343091947$

\end{document}